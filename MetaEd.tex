\documentclass[
    ngerman,american
    ]{scrartcl}

    % ##########################################

    \usepackage{babel}
    \usepackage[utf8]{inputenc} 
    \usepackage{csquotes}
    \usepackage{enumitem}
    \usepackage{ifthen}
    \usepackage{lipsum}
    
    
\newcommand{\sectionIntroduction}
{
   {\section{Introduction}}{}
}

\newcommand{\sectionImpact}
{
   {\section{Significance and impact of the research}}{}
}

\newcommand{\sectionImpactDescription}
{
   {In this section, the significance and impact of the proposed work is described.}{}
}

\newcommand{\sectionMethod}
{
    \section{Research Methodology}{}
}

\newcommand{\sectionResources}
{
    {\section{Resources and Facilities}}{}
}

\newcommand{\sectionSource}
{
    {\section{Relevant Related Work}}{}
}

\newcommand{\phaseOne}
{
    {Conduct literature review}{}
}

\newcommand{\phaseTwo}
{
    {Design and architect the MetaEd platform}{}
}

\newcommand{\phaseThree}
{
    {Build the MetaEd platform with AI support}{}
}

\newcommand{\phaseFour}
{
    {Deploy the MetaEd platform in large-scale}{}
}

    \usepackage[
        bibencoding=utf8, 
        style=alphabetic
    ]{biblatex}

    \bibliography{MetaEd}
    
    
    \usepackage{amsmath}
    \title{
         Metaverse and Artificial Intelligence for Next Generation Educational Technology Platforms        
    }
    
    % ##########################################
    
    \author {
        Thanh Vo and Sam Goundar   
    }
    
    % ##########################################
    
    \begin{document}
      \maketitle
        \begin{abstract}
         
         This project uses Metaverse and Artificial Intelligence to develop an educational technology
	platform and training programs that are challenging to offer in real-world setting. 
	MetaEducation has the potential to radically transform the teaching and learning landscape. It’s
	power, though rudimentary is already realised with the use of VR, AR, XR, and MR in existing
	educational platforms. It has been cheaper, easier, and safer to provide STEM education using
	these, rather than risking training in real-life scenarios. Trainee neurosurgeons and pilots are a
	good example. Artificial Intelligence will ensure that the Meta-Education platform follows the
	rules prescribed by the Teacher. Artificial Intelligence is also the enabler of simulation based
	STEM training. For learners to be able to get the feel of training in the authentic world, Artificial
	Intelligence is needed to ensure learners are able to work and learn with intelligent NPC tutors,
	peers, and other learners.

        \end{abstract}
        
        % ##########################################       
        
        \sectionIntroduction
        Many modern technologies come to life from science fictions. Metaverse is one of such new technologies. 
        The term was firstly coined in the novel Snow Crash in 1992 
        and the Metaverse concepts were described more vividly in great details in the movie Ready Player One 
        in 2018 by the famous director Steven Spielberg \cite{mystakidis_meta_encyclopedia_2022} \cite{park_meta_taxonomy_2022}. 
        Metaverse has become more and more available for everyone in our daily activities. More than 60\% American teenagers play Roblox, 
        an online video game in which gamers interact with the virtual environment and with other people via 
        their avatars. The recent investments from big tech companies such as Meta, new brand name of Facebook Inc, 
        draw more attention from the Computer Science research community.
        \par
        However, the research on applications of Metaverse, especially in Education, is still in its infancy 
        \cite{bokyung_meta_ed_2022} \cite{hwang_meta_ai_2022} \cite{tlili_metaed_2022}. Following 
        the emergence of Metaverse, we are in need of a complete educational platform which could provide immersive 
        learning experience to the students. Once such platforms are available for billions of learners on daily basis, the platforms 
        should be self-operation with minimum human monitoring effort. The learners should be able to access
        the platforms any time and anywhere with as real as possible experience. We can learn and find the solutions 
        to solve this problem from the gaming industry. With the support of AI technologies, Massive Multiplayer Online Games 
        like Warcraft or Smite populate army of bots which are indistinguishable from the real players.  AI technologies 
        seem to be the only solution to provide intelligent NPC (non-player characters) teachers, students in the virtual 3D classrooms.         
        \par
        This research project will explore the possibilities to apply the combination of Metaverse and Artificial Intelligence 
        to build the next generation educational technology platforms.                
        
        % ##########################################
                       
        \sectionImpact        
        
          In developing countries, millions of students desire to go aboard to get
         the high-quality education from the developed Western world. RMIT and many international 
         universities have contributed significantly to realize the dreams of the young generations by establishing 
         many campuses with modern infrastructures and facilities. The students also benefit from world-class 
         teaching staffs with great experience working at top-notch education institutions. However, such opportunities 
         are only available for students from rich families who are able to support 3-4 years studying on-campus. 
         The project would create the virtual reality of the educational environment of top universities which are 
         available and affordable for students and learners from the less developed areas.
         \par
         The MetaEd platform would significantly reduce the cost of risking trainings such as pilot trainings.
        
        % ##########################################
    
    	\sectionMethod
	The research project will be realized through four phases.
	\begin{description}{}
	\item [\phaseOne] 
	In the first phase, we will conduct a literature review on the fields of AI and Metaverse to find out all possible solutions
	of AI techniques which can be applied in Metaverse to create intelligent NPC tutors and learners.
	
	\item [\phaseTwo]
	In the second phase, we will propose the architecture for the MetaEd platform. We aim to build a large-scale, highly secure 
	and easily maintainable which can be used by billions of learners and educators daily.  
	
	\item [\phaseThree]
	In the third phase, we will implement the platform based on the architecture built in the second phase and the collected techniques 
	or innovative methods applying AI to simulate the real classrooms with intelligent NPC tutors and learners.
	
	\item [\phaseFour]
	In the final phase, after the whole platform is completed, we will deploy the platforms for universities in Vietnam and Australia.
	Then, we will perform an empirical study to show the impact of the platform on improving the education quality in both developing
	and developed areas. 
	
	\end{description}{}	
	
	% ##########################################
	
	\sectionResources
	\sectionResourcesDescription
	The research project requires intensive use of VR, AR, XR, and MR devices. The researchers need to be equipped with latest 
	devices to be able to apply all new features of the hardware technology. The researchers need to interact and get support frequently 
	with the participants of the classes and courses that would apply Metaverse in their teaching.
      
        % ##########################################

        \sectionSource
        \sectionSourceDescription
        Some research papers raise the open
        research questions for this topic. Some research built a simple prototype and proposed simple architecture for Metaverse of one university campus. There is a need
        to build complete an educational platform using Metaverse. In such platforms, we need to provide an virtual environment in which users can interact with 
        the platform and have as real as possible learning experience. 

        % ##########################################
        % # Overview of identified relevant work
        % ######
        % The goal of this section is to provide an overview of the relevant and significant 
        % related work identified so far. Make sure that your cited sources are of appropriate
        % quality!
        % 
        % Please include:
        %  - a citation of the source using Latex facilities (incl. a generated list of 
        %    references)
        %  - a brief descriptions of the source and a statement why this is relevant for 
        %    your work (1-2 sentences)
        % 
        \begin{description}
        \item[\cite{park_meta_taxonomy_2022}] insert brief description
        \item[\cite{hwang_meta_ai_2022}] insert brief description
        \item[\cite{bokyung_meta_ed_2022}]
        \item[\cite{mystakidis_meta_encyclopedia_2022}]
        \item[\cite{tlili_metaed_2022}]
        \item[\cite{duan_metaed_2021}]
        \end{description}
        % ##########################################
      	\printbibliography
    \end{document}