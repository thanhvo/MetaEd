\documentclass[
    ngerman,american
    ]{scrartcl}

    % ##########################################

    \usepackage{babel}
    \usepackage[utf8]{inputenc} 
    \usepackage{csquotes}
    \usepackage{enumitem}
    \usepackage{ifthen}
    \usepackage{lipsum}
    
    
\newcommand{\sectionIntroduction}
{
   {\section{Introduction}}{}
}

\newcommand{\sectionImpact}
{
   {\section{Significance and impact of the research}}{}
}

\newcommand{\sectionImpactDescription}
{
   {In this section, the significance and impact of the proposed work is described.}{}
}

\newcommand{\sectionMethod}
{
    \section{Research Methodology}{}
}

\newcommand{\sectionResources}
{
    {\section{Resources and Facilities}}{}
}

\newcommand{\sectionSource}
{
    {\section{Relevant Related Work}}{}
}

\newcommand{\phaseOne}
{
    {Conduct literature review}{}
}

\newcommand{\phaseTwo}
{
    {Design and architect the MetaEd platform}{}
}

\newcommand{\phaseThree}
{
    {Build the MetaEd platform with AI support}{}
}

\newcommand{\phaseFour}
{
    {Deploy the MetaEd platform in large-scale}{}
}

    \usepackage[
        bibencoding=utf8, 
        style=alphabetic
    ]{biblatex}

    \bibliography{MetaEd}
    
    
    \usepackage{amsmath}
    \title{
         Metaverse and Artificial Intelligence for Next Generation Educational Technology Platforms        
    }
    
    % ##########################################
    
    \author {
        Thanh Vo and Sam Goundar   
    }
    
    % ##########################################
    
    \begin{document}
      \maketitle
        \begin{abstract}
         
         This project uses Metaverse and Artificial Intelligence to develop an educational technology
	platform and training programs that are challenging to offer in real-world setting. 
	MetaEducation has the potential to radically transform the teaching and learning landscape. It’s
	power, though rudimentary is already realised with the use of VR, AR, XR, and MR in existing
	educational platforms. It has been cheaper, easier, and safer to provide STEM education using
	these, rather than risking training in real-life scenarios. Trainee neurosurgeons and pilots are a
	good example. Artificial Intelligence will ensure that the Meta-Education platform follows the
	rules prescribed by the Teacher. Artificial Intelligence is also the enabler of simulation based
	STEM training. For learners to be able to get the feel of training in the authentic world, Artificial
	Intelligence is needed to ensure learners are able to work and learn with intelligent NPC tutors,
	peers, and other learners.

        \end{abstract}
        
        % ##########################################       
        
        \sectionIntroduction
        Many modern technologies come to life from science fictions. Metaverse is one of such new technologies. 
        The term was firstly coined in the novel Snow Crash in 1992 
        and the Metaverse concepts were described more vividly in great details in the movie Ready Player One 
        in 2018 by the famous director Steven Spielberg \cite{mystakidis_meta_encyclopedia_2022} \cite{park_meta_taxonomy_2022}. 
        Metaverse has become more and more available for everyone in our daily activities. More than 60\% American teenagers play Roblox, 
        an online video game in which gamers interact with the virtual environment and with other people via 
        their avatars. The recent investments from big tech companies such as Meta, new brand name of Facebook Inc, 
        draw more attention from the Computer Science research community.
        \par
        Nowadays, 2D Learning Environments are so popular and accessible with the developments of Open Education, Massive Open Online Courses which
        depend on Web applications. However, such environments have many limitations
        such as low-self perception, no presence and inactivity of learners, instructors and crude emotional expression  \cite{mystakidis_meta_encyclopedia_2022}. 
        Although the development of Metaverse in Education is still in its infancy, it shows much potential to overcome those issues
        with 3D immersive spatial environments. 
        Thus, we are in need of a complete Metaverse educational platform.
        Once such platforms are available for billions of learners on daily basis, the platforms 
        should be self-operation with minimum human monitoring effort. The learners should be able to access
        the platforms any time and anywhere with as real as possible experience. We can learn and find the solutions 
        to solve this problem from the gaming industry. With the support of AI technologies, Massive Multiplayer Online Games 
        like Warcraft or Smite, populate army of bots which are indistinguishable from the real players.  AI technologies 
        seem to be the only solution to provide intelligent NPC (non-player characters) teachers, students in the virtual 3D classrooms.         
        \par
        This research project will explore the possibilities to apply the combination of Metaverse and Artificial Intelligence 
        to build the next generation educational technology platforms.                
        
        % ##########################################
                       
        \sectionImpact        
         
         The impact of the Metaverse in education is positive and effective in different situations\cite{tlili_metaed_2022}. 
         MetaEd platforms have shown many advantages over 2D online Learning Environment. With immersive 3D learning environments, 
         the students can feel more present, can express more emotions, pay more attention to lectures, are more engaged and interested 
         in learning. By leveraging the characteristics of the Meteverse, educators can
         design learning activities that favor students' freedom and experience \cite{bokyung_meta_ed_2022}. 
         In Meta Education, students are co-owners of virtual spaces and co-creators of personalized curricula. 
         Moreover, Metaverse technology is applied to simulate learning scenarios which are very dangerous and costly in real-life.   
         The MetaEd platform would significantly reduce the cost of flight training, aircraft maintainance training and surgical training. 
         Metaverse is applied to provide safety training for children in outdoor environment with VR Kinect sensor and 
         Unity game engine \cite{park_meta_taxonomy_2022}.
         Metaverse is in the early stages of its development. 
         When the technology is mature, along with Web 4.0, it would completely replace the currently dominant online learning platforms.          
         \par
         In developing countries, millions of students desire to go aboard to get
         the high-quality education from the developed regions. Many top international 
         universities like RMIT have contributed significantly to realize the dreams of the young generations by establishing 
         many offshore campuses with modern infrastructures and facilities in developing regions. The students also benefit from world-class 
         teaching staffs with great experience working at top-notch educational institutions. However, such opportunities 
         are only available for students from rich families who are able to support 3-4 years studying on-campus. 
         One goal of the project is to create the mirror world of the educational environment at top universities which are 
         available and affordable for students and learners from the less developed areas.
         
        % ##########################################
    
    	\sectionMethod
	The research project will be realized through four phases.
	\begin{description}{}
	\item [\phaseOne] 
	In the first phase, we will conduct a literature review on the fields of AI and Metaverse to find out all possible AI techniques
	which can be applied in Metaverse to create intelligent NPC tutors and learners.
	
	\item [\phaseTwo]
	In the second phase, we will propose the architecture for the MetaEd platform. We aim to build a large-scale, highly secure 
	and easily maintainable which can be used by billions of learners and educators daily.  
	
	\item [\phaseThree]
	In the third phase, we will implement the platform based on the architecture proposed in the second phase and the collective techniques 
	or innovative methods applying AI to simulate the real classrooms with intelligent NPC tutors and learners.
	
	\item [\phaseFour]
	In the final phase, after the whole platform is completed, we will deploy the platforms for universities in Vietnam and Australia.
	Then, we will perform an empirical study to evaluate the impact of the platform on improving the education quality in both developing
	and developed areas. 
	
	\end{description}{}	
	
	% ##########################################
	
	\sectionResources
	The research project requires intensive use of Virtual Reality, Augmented Reality, Extended Reality, and Mixed Reality devices. 
	The researchers need to be equipped with latest devices to be able to apply all new features of the hardware technology. 
	The researchers need to interact and get support frequently 
	with the participants of the classes and courses that would apply Metaverse in their teaching and learning.
      
        % ##########################################

        \sectionSource
        
        In the recent years, many researchers have conducted some early impressive work applying Metaverse in Education. 
        The research group at The Chinese University of Hong Kong, Shenzhen \cite{duan_metaed_2021} have built a Metaverse of their campus as
        a mobile application based on Unity game engine. They proposed simplified three-layer metaverse architecture including infrastructure, interaction
        and ecosystem.  The research group from National Taiwan University of Science and Technology \cite{hwang_meta_ai_2022} 
        provided definitions, roles and identified the potential research issues for MetaEd from AI perspective. 
        The research group from University of Polytechnique Hauts-de-France \cite{tlili_metaed_2022} sought for a roadmap of future research directions of MetaEd.  
        Our research project would like to investigate the following research topics raised in the articles.
        \begin{enumerate}{}
        \item Develope metaverse-based educational platforms.
        \item Leverage AI technologies to analyze students' behaviors and interaction patterns.
        \item Apply blockchain technology to provide security of MetaEd platforms. 
        \end{enumerate}{}
        

        % ##########################################
      	
	\printbibliography
    \end{document}